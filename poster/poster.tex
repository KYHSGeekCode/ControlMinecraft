% Copyright (c) 2013 Joost van Zwieten

% Permission is hereby granted, free of charge, to any person obtaining a copy
% of this software and associated documentation files (the "Software"), to deal
% in the Software without restriction, including without limitation the rights
% to use, copy, modify, merge, publish, distribute, sublicense, and/or sell
% copies of the Software, and to permit persons to whom the Software is
% furnished to do so, subject to the following conditions:

% The above copyright notice and this permission notice shall be included in
% all copies or substantial portions of the Software.

% THE SOFTWARE IS PROVIDED "AS IS", WITHOUT WARRANTY OF ANY KIND, EXPRESS OR
% IMPLIED, INCLUDING BUT NOT LIMITED TO THE WARRANTIES OF MERCHANTABILITY,
% FITNESS FOR A PARTICULAR PURPOSE AND NONINFRINGEMENT. IN NO EVENT SHALL THE
% AUTHORS OR COPYRIGHT HOLDERS BE LIABLE FOR ANY CLAIM, DAMAGES OR OTHER
% LIABILITY, WHETHER IN AN ACTION OF CONTRACT, TORT OR OTHERWISE, ARISING FROM,
% OUT OF OR IN CONNECTION WITH THE SOFTWARE OR THE USE OR OTHER DEALINGS IN
% THE SOFTWARE.
%
\documentclass{tudelftposter}

% optional, makes QR code clickable
\usepackage[implicit=false,bookmarks=false]{hyperref}
\usepackage[utf8]{inputenc} % allow utf-8 input
\usepackage[T1]{fontenc}    % use 8-bit T1 fonts
\usepackage{url}            % simple URL typesetting
\usepackage{booktabs}       % professional-quality tables
\usepackage{amsfonts}       % blackboard math symbols
\usepackage{nicefrac}       % compact symbols for 1/2, etc.
\usepackage{microtype}      % microtypography
\usepackage{lipsum}
\usepackage{listings}
\usepackage{amsmath}
\usepackage{diagbox}
\usepackage{tabularx}
\usepackage{longtable}
\lstset{language = C}
\usepackage{graphicx,changepage}
\usepackage{float}
\usepackage{graphicx}
\usepackage{kotex}
\usepackage{xcolor}
\usepackage{color}
\usepackage{caption}
\usepackage{minted}
\usepackage{svg}
\usepackage{cclicenses}


\usepackage{setspace}

\hypersetup{%
  colorlinks=true,% hyperlinks will be black
  linkbordercolor=red,% hyperlink borders will be red
  pdfborderstyle={/S/U/W 2}% border style will be underline of width 1pt
}


\doublespacing
\graphicspath{./img/}
\title{윈도우즈 시스템에서 집중방해요소 차단 기술\\
\large 네트워크 패킷(packet) 차단과 프로세스 제어}

\addauthornote{mail}[@]{\ttfamily yhs0602@snu.ac.kr}
% \addauthornote{diam}{Delft Institute of Applied Mathematics, TU Delft}
% \addauthornote{tp}{Department of Theoretical Physics, TU Delft}

% \addauthor[mail,diam]{J.F. Nash}
% \addauthor[diam,tp]{C.F. Gau{\ss}}
\addauthor{2019-13674 양현서}
% \addauthor[tp]{R.P. Feynman}

% \addfootimage(c:right column.center)[Delft Institute of Applied Mathematics]{tudelft}
% \ifqrcodessupported
%   % NOTE: compile this document with lualatex instead of pdflatex
%   \addfootqrcode(l:left column.left)[tudelft-poster repository]{http://github.com/joostvanzwieten/tudelft-poster}
% \else
%   \addfootobject*(l:left column.left)[research web page]{%
%     \begin{tikzpicture}
%       \node[red,draw,inner sep=1ex] {%
%         \begin{minipage}{15cm}
%           \footnotesize%
%           Please compile this document with lualatex to see a QR code here:\\
%           \null\texttt{\quad lualatex \jobname}
%         \end{minipage}};
%     \end{tikzpicture}}
% \fi

\begin{document}

% \SetSerifFonts{gt}{gt}
% \SetSansFonts{mj}{mj}
\section{서론}
\subsection{연구 주제}
윈도우 환경에서 온라인 학습 집중 보조를 위해 필요한 네트워크 패킷(packet) 차단과 프로세스 제어 기술을 연구한다.

\subsection{연구 동기 및 목적}
\begin{enumerate}
  \item 온라인 수업이 증가하면서 집중을 도와주는 솔루션의 개발이 절실하나, 소수의 솔루션만이 제공되어 소비자의 선택의 폭이 좁다.
  \item 집중방해요소 차단 솔루션을 구현하려면 특수 권한 접근을 위해 NT 레거시 드라이버 개발이 필수적이지만 이를 구현하기 위한 기술인 DDK(Driver Development Kit)는 진입 장벽이 높다.
\end{enumerate}
이에 따라 집중방해요소 차단 솔루션을 제작할 때 필요한 부분을 집중적으로 연구하는 것이 이 연구의 목적이다.

\section{이론적 배경}
\subsection{API(Application Programming Inverface)}
API란 응용프로그램이 특별한 기능을 사용할 수 있도록 운영체제가 제공하는 기능의 집합이다.
\subsection{보호 링과 윈도우 드라이버 모델}
% \begin{figure}
%   \centering
%   \includegraphics[width=.23\textwidth]{Priv_rings}
%   \caption{특권 레벨이 높은 작업을 하기 위해서는 디바이스 드라이버가 필요하다. \bysa}
%   \label{fig:test}
% \end{figure}
높은 특권을 가진 링 0에서 동작하는 프로그램을 개발하기 위해서는 DDK라는 특별한 API를 사용하여야 하며, 윈도우 드라이버 모델이 이러한 API를 제공한다.

\subsection{WFP(Windows Filtering Platform)}
WFP는 네트워크 필터링 기능을 제공하는 응용 프로그램을 만들기 쉽도록 윈도우가 제공하는 API의 모음이다.

\subsection{콜백함수}
콜백함수란 특정 사건이 발생했을 때 실행하게 설정하는 개발자가 정의한 코드이다.

\section{연구 방법}
\begin{enumerate}
  \item 윈도우 디바이스 드라이버를 개발하는 방법 연구 : 윈도우 디바이스 드라이버를 개발하기 위한 지식이 담긴 도서(윈도우 디바이스 드라이버)를 조사하여 정리한다. 기본적인 기능을 가진 윈도우 디바이스 드라이버를 생성하여 제어하는 방법을 실험하고 정리한다.
  \item 방화벽 개발 관련 정보를 참고하여 네트워크 필터링 방법 연구: 네트워크 패킷(packet) 필터링을 구현한 라이브러리를 조사하여 구현을 분석하거나 해당 라이브러리를 이용하는 방법을 연구하고 테스트한다.
  \item 백신 개발 관련 정보를 참고하여 프로세스 필터링 방법 연구: 오픈 소스 백신 소프트웨어의 소소 크도나 Windows API 문서를 참조하여 프로세스 필터링 프로그램을 개발하여 테스트한다.
\end{enumerate}

\section{연구 결과}
연구 결과, 다음과 같이 구성하여 개발하면 네트워크 패킷 차단 및 프로세스 제어 기능을 구현할 수 있다는 결론을 내려 예시 응용 프로그램을 개발하였다.
\subsection{구조도}
% \begin{figure}
%   \centering
%   \includegraphics[width=.3\textwidth]{result_revised}
%   \caption{연구 결과 시스템 개념도.}
%   \label{fig:result}
% \end{figure}

\begin{enumerate}
  \item \href{https://docs.microsoft.com/en-us/windows/win32/api/ioapiset/nf-ioapiset-deviceiocontrol}{DeviceIoControl} (마셜링 이용) 프로세스 차단 목록 관리
  \item \href{https://docs.microsoft.com/en-us/windows-hardware/drivers/ddi/ntddk/nf-ntddk-pssetcreateprocessnotifyroutine}{PsSetCreateProcessNotifyRoutine}으로 콜백 함수 등록
  \item 프로세스 실행 시 운영체제가 콜백 함수 호출
  \item 드라이버가 차단 목록에 의해 프로세스 실행 허용/차단
  \item 네트워크 감시 기능 실행/중지, 차단 사이트 관리
  \item \href{https://reqrypt.org/windivert.html}{WinDivert} 라이브러리 이용
  \item WinDivert 드라이버 로드 및 제어
  \item \href{https://docs.microsoft.com/ko-kr/windows/win32/fwp/windows-filtering-platform-start-page}{Windows 필터링 플랫폼}에 WinDivert.sys 드라이버 등록
  \item 네이티브 네트워크 API 호출
\end{enumerate}

다음은 프로그램 실행 결과이다. 예시의 의도대로 메모장 실행이 차단되는 것을 볼 수 있다.

% \begin{figure}
%   \centering
%   \includegraphics[width=.35\textwidth]{block}
%   \caption{디바이스 드라이버가 메모장 실행을 차단한 모습. 메모장 프로세스는 생성되지 않으며, 메모장 실행이 차단되었다는 로그가 출력된다.}
%   \label{fig:block}
% \end{figure}

\subsection{연구 결과 소스 코드 저장소}
\href{https://github.com/KYHSGeekCode/Self-Protecting-Driver}{Github}에서 본 포스터의 소스 코드를 탐색할 수 있다. 해당 저장소의 주소는 다음과 같다.
\href{https://github.com/KYHSGeekCode/Self-Protecting-Driver}{https://github.com/KYHSGeekCode/Self-Protecting-Driver}
% \subsection{네트워크 패킷 차단}
% 네트워크 패킷을 중간에서 감시하기 위해서 IP Filter driver를 후킹하는 방법이 있다. WinDivert 라이브러리가 해당 기능을 구현하였다. Windows Filtering Platform (WFP)를 이용하여 
% FwpmSubLayerAdd0
% \subsection{프로세스 생성 제어}
% \subsubsection{콜백함수 등록}
% \lstinline{PsSetCreateProcessNotifyRoutineEx} 함수를 이용하면 윈도우즈 시스템 내에서 프로세스가 생성될 때 실행할 콜백함수를 설정할 수 있다. \lstinline{DriverEntty}에서 호출되는 \lstinline{InstallProcessProtect} 함수에서는 이 함수를 이용하여 콜백함수를 등록한다.
% \lstinline{대상 프로세스 탐지}
% \lstinline[breaklines=true]{VOID ProcessNotifyCallbackEx(PEPROCESS  Process, HANDLE  ProcessId,	PPS_CREATE_NOTIFY_INFO CreateInfo)} 함수에서는 등록된 프로세스 경로 패턴과 현재 실행되려하는 프로세스의 경로를 비교하여 차단과 허용 여부를 결정한다. 이 떄 \lstinline{FsRtlIsNameInExpression}함수를 이용한다. 이 함수를 사용할 때의 주의점은 ignorecase일 경우 패턴이 대문자여야 한다는 것이다.
% 이제 차단하거나 허용할 프로세스 리스트를 관리하는 것에 초점을 맞추어 보자. 프로그램 실행 중에 언제든지 차단/허용 목록이 추가되거나 제거될 수 있다. 따라서 리스트 자료구조를 사용한다. 커널 레벨에서 동적 메모리 할당은 까다로운 작업이지만, ExAllocatePoolZero함수를 이용하여 메모리를 할당받을 수 있다. IRQL이 낮은 레벨인가 아무튼 잘 된다. 이 메모리를 해제할 때는 ExFreePool을 이용하면 된다. 또한 \lstinline{CONTAINING_RECORD} 사용법도 눈여겨보면 좋을 것이다.

\section{한계점 및 논의}
\subsection{HTTPS}
HTTPS 환경에서는 패킷이 암호화되기 때문에 패킷을 감시하는 것만으로는 사용자가 접근하는 URL정보를 얻어오기 어렵다. 이 문제를 해결한다면 한 단계 더 나아갈 수 있을 것이다.

\section{시사점}
이 연구는 타 분야에 비해 자료를 이해하기 어려운 DDK를 이용한 NT 레거시 윈도우 디바이스 개발이라는 분야에 유용한 예시를 제공하며, 이 연구를 통해 작성한 구현을 이용하여 비대면 강의 시 학생들의 집중을 도와주는 솔루션을 만드는 데 큰 도움을 준다는 데서 의의가 있다.
\bibliographystyle{unsrt}
\begin{thebibliography}{1}
    \bibitem{WDD2009}
    이봉석, 『윈도우 디바이스 드라이버』, {\em 한빛미디어}, 2009.
\end{thebibliography}


\end{document}
% \begin{equation}
  %   \int_\Omega v_i \left(
  %     u_{i,t} + u_j u_{i,j} + p_{,i} - R^{-1} u_{i,jj}
  %   \right) =
  %   \int_\Omega v_i f_i.
  % \end{equation}
  % \experimentalblockright{%
%   SPAM:
%   Nulla malesuada porttitor diam. Donec felis erat, congue non, volutpat at,
%   tincidunt tristique, libero.}

% vim: tw=80:ts=2:sts=2:sw=2:et:fdm=marker:fmr=[[[,]]]

